\chapter{Materiales y Métodos}
\lipsum[1-2]
\begin{table}[!ht]
%\renewcommand{\arraystretch}{1.5}
%\setlength{\extrarowheight}{10pt}
\centering
\caption[Fossils used as calibration points]{Fossils used as calibration points and their specified parametes. Tomado de \citet{couvreur2011}}
\label{tabla:fosiles}
\begin{tabu} to 0.9\textwidth { X[,m,l] X[m,c] X[m,c] X[m,c] }
\toprule
\cabeza{Fossil name} & \cabeza{Hard lower bound (Ma)} & \cabeza{Soft upper bound 95\% (Ma)} & \cabeza{Exponential mean (uncertainty)}\\
\midrule
\textit{Sabalites carolinensis} & 84.0 & 90.0 & 2.0 \\
\midrule
Mauritiidites & 41.2 & 65.2 & 8.0 \\
\midrule
Attaleinae & 22.5 & 31.5 & 3.0 \\
\midrule
\textit{Hyphaene kapelmanii} & 26.0 & 33.5 & 2.5 \\
\bottomrule
\multicolumn{4}{l}{\footnotesize Ma = million years ago}\\
\end{tabu}
\end{table}
\lipsum[3-5]

Esta es una referencia a la tabla de puntos de calibración con fósiles (tabla~\ref{tabla:fosiles})